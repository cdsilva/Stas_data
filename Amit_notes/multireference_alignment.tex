\documentclass[12pt]{article}
\usepackage{amssymb}
\usepackage{amsmath}
\usepackage{dsfont}

\title{Multireference Alignment Notes}

\begin{document}

\maketitle

We are given images $I_1, \dots, I_n$.

We want to find rotations $R_1, \dots, R_n \in SO(2)$ such that the images are optimally aligned.

We define a cost function $c(R_i I_i, R_j I_j)$.
%
We would like to solve the optimization problem
\begin{equation} \label{eq:opt1}
\min_{R_1, \dots R_n \in SO(2)} \sum_{i \ne j} c(R_i I_i, R_j I_j)
\end{equation}

Since the cost is invariant to global rotations, we can rewrite \eqref{eq:opt1} as
\begin{equation}
\min_{R_1, \dots R_n \in SO(2)} \sum_{i \ne j} c(I_i, R_i^T R_j I_j)
\end{equation}

We discretize the space of rotations into $L$ intervals, and write each rotation as a number $l_i \in \mathbb{Z}_L, i=1, \dots, n$. 
%
We also define a cost function between a {\em specific} pair of images $I_i$ and $I_j$ as $c_{ij}(l_i - l_j)$; 
this cost function is only a function of the difference in rotations, $l_i - l_j$. 
%
we rewrite the optimization problem as
$$\min_{l_1, \dots, l_n \in \mathbb{Z}_L} \sum_{i \ne j} c_{ij} (l_i - l_j)$$

We now note that each rotation $l_i-l_j \in \mathbb{Z}_L$ maps to a {\em cyclic} permutation matrix $\pi_{ij} \in \mathbb{R}^{L \times L}$, because each rotation can be thought of as a cyclic permutation of the $L$ ``ticks'' on the unit circle.
%
We will optimize over the space of cyclic permutation matrices, rather than the space of rotation matrices.

However, the set of cyclic permutation matrices is non-convex (e.g. because the average of two permutations matrices is not a permutation matrix). 
%
The convex hull of permutation matrices can be shown to be the set of all doubly stochastic matrices, which satisfy the following constraints
\begin{equation} \label{eq:doublestoch}
\begin{aligned}
\pi_{ij}(k,l) &  \ge 0 \forall k, l \\
\pi_{ij} \mathds{1} & = \mathds{1}\\
\mathds{1}^T \pi_{ij} & = \mathds{1}^T 
\end{aligned}
\end{equation}

We also require the cyclic condition to hold:
\begin{equation} \label{eq:cyclic_const}
\pi_{ij}(k,l) = \pi_{ij}(k+1, l+1)
\end{equation}

We would also like {\em consistency} relations to hold, e.g. $\pi_{ij} \pi_{jk} = \pi_{ik}$.
%
This is trivially satisfied if $\exists \pi_1, \dots, \pi_n$ such that $\pi_{ij} = \pi_i^T \pi_j \forall i, j$. 
%
Therefore, we want
\begin{equation} \label{eq:Gram_const}
\Pi = 
\begin{bmatrix}
\pi_{11} & \pi_{12} & \cdots & \pi_{1n} \\
\pi_{21} & \pi_{22} & \cdots  & \pi_{2n} \\
\vdots & \vdots & \ddots & \vdots \\
\pi_{n1} & \pi_{n2} & \cdots  & \pi_{nn} 
\end{bmatrix} = 
\begin{bmatrix}
\pi_1^T \\
\vdots \\
\pi_n^T 
\end{bmatrix}
\begin{bmatrix}
\pi_1^T \cdots \pi_n^T 
\end{bmatrix}
\end{equation}
where $\Pi \in \mathbb{R}^{nL \times nL}$ is the Gram matrix, where the $i,j$ block of $\Pi$ is $\pi_{ij} \in \mathbb{R}^{L \times L}$.

\eqref{eq:Gram_const} holds if $rank(\Pi) = L$ and $\Pi$ is positive semidefinite. 
%
We relax the constrant that $rank(\Pi) = L$, and only require  
\begin{equation} \label{eq:psd_const}
\Pi \succeq 0
\end{equation}

We also require that
\begin{equation} \label{eq:identity_const}
\pi_{ii} = I_{L \times L}
\end{equation}

We then construct the matrices $C_{ij}  \in \mathbb{R}^{L \times L}$ for $i, j= 1, \dots, n$, where
\begin{equation}
C_{ij}(l_1, l_2) = c_{ij}(l_1 - l_2)
\end{equation}

The objective function can then be written as
\begin{equation} \label{eq:one_cost}
\min_{\pi_{ij}} \sum_{i \ne j} Tr(C_{ij} \pi_{ij})
\end{equation}

We then define the matrix $C \in \mathbb{R}^{nL \times nL}$, where
$$C = \begin{bmatrix}
 C_{11} & C_{12} & \cdots & C_{1n} \\
 C_{21} & C_{22} & \cdots & C_{2n} \\
\vdots & \vdots & \ddots & \vdots \\ 
C_{n1} & C_{n2} & \cdots & C_{nn} \\
\end{bmatrix} $$

We can then write \eqref{eq:one_cost} as 
\begin{equation} \label{eq:total_cost}
\min_{\Pi} Tr(C^T \Pi)
\end{equation}

\eqref{eq:total_cost}, togther with constraints \eqref{eq:doublestoch}, \eqref{eq:cyclic_const}, \eqref{eq:psd_const}, and \eqref{eq:identity_const}, define the semidefinite program that we would like to solve. 

\end{document}
