% irt.tex
% documentation for image reconstruction toolbox

\documentclass{article}
\usepackage{jf-all}
\usepackage[dvips,text={7in,9in}]{geometry}

\newcommand{\ty}[1] {\jfverbtt{#1}}
\newcommand{\matlab} {\textsc{Matlab}\xspace}
\newcommand{\irt} {IRT\xspace}
\newcommand{\fatrix} {\texttt{Fatrix}\xspace}
\newcommand{\fatrixx} {\texttt{fatrix2}\xspace}

\newcommand{\A} {\blmath{A}}
\newcommand{\x} {\blmath{x}}
\newcommand{\y} {\blmath{y}}
\newcommand{\z} {\blmath{z}}
\newcommand{\pn} {\jfunl{\vec{n}}}
\newcommand{\nd} {\xmath{n_{\mathrm{d}}}}
\newcommand{\np} {\xmath{n_{\mathrm{p}}}}
\newcommand{\zumnm} {\sum_{n=0}^{N-1} \sum_{m=0}^{M-1}}

\newlength{\figwidth} % for variable-size figures
\newcommand{\infig}[3][\fdir]{ % [dir] file, size
\setlength{\figwidth}{#3\linewidth}%
\input{#1/#2}%
}

\begin{document}

\title{
Image Reconstruction Toolbox
for \matlab
(and Freemat)
}
\author{Jeffrey A. Fessler\\
University of Michigan
\\
\ty{fessler@umich.edu}
\date{\today}
}

\maketitle

\section{Introduction}

This is an initial attempt at documentation
of the image reconstruction toolbox (\irt)
for \matlab,
and any other \matlab emulator
that is sufficiently complete.
This documentation is, and will always be,
hopelessly incomplete.
The number of options and features
in this toolbox is ever growing.


\section{Overview}

\input s,over

\newpage
\input s,mask

\newpage
\section{Special structures}

The \irt uses
some special custom-made object classes
extensively:
the \ty{strum} class,
which provides structures with methods,
and
the \fatrixx class
(and its obsolete predecessor, the \fatrix class),
that provide a ``fake matrix'' object.
These objects exploit
\matlab's object oriented features,
specifically operator overloading.
The following overview of these objects
should help in
understanding the reconstruction code.

\input s,strum

\input s,fatrix2

\newpage
\input s,fatrix

\newpage
\input s,other

\end{document}
