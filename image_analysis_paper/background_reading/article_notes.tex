\documentclass[12pt]{article}


\usepackage[backend=bibtex]{biblatex}
\usepackage[left=1in, right=1in, top=1in, bottom=1in]{geometry}

\bibliography{references,../../../references/references}

\begin{document}

\section{References from Stas}

\begin{enumerate}

\item \fullcite{wei2006conditional}
\begin{itemize}
		\item Want to develop a more informative growth chart metric
		\item Instead of looking at how atypical a current measurement is for an individual, they look at how atypical a time series is for an individual 
		\item This means that a person will be flagged if, for example, their weight changes dramatically in a short period of time, even if both measurements are within the ``normal'' range
		\item Fit model to data for ``normal'' time--series behavior
\end{itemize}

\item \fullcite{wei2006quantile}
\begin{itemize}
	\item Develop nonparameteric models (using spline regression) for growth charts
	\item Typical growth charts rely on normality assumptions (data is often transformed to fulfill such assumptions)
	\item Using nonparametric models can allow them to fit a much broader range of behaviors (i.e. bimodal distributions)
\end{itemize}

\item \fullcite{li2009stochastic}
\begin{itemize}
	\item Want to look at mechanism for {\em eve} stripe formation
	\item Propose a large gene/protein interaction network
	\item Use maximum likelihood to prune the network and find the most probable network that would give rise to the observed spatiotemporal dynamics
\end{itemize}

\item \fullcite{jordan2013behavioral}
\begin{itemize}
	\item Look at phenotypic diversity of the swimming of the ciliate {\em Tetrahymena}
	\item Record the swimming trajectories of many samples over their lifetimes
	\item Define similarity metric between trajectories (compare distributions of speeds and angular velocities)
	\item Use multidimensional scaling (MDS) to embed the trajectories in a lower-dimensional space
	\item Find that there are two ``important'' dimensions that describe all trajectories
	\item Can then look at whether these (low-dimensional) phenotypes are preserved through generations
\end{itemize}

\item \fullcite{tucker2010pseudotemporal}
\begin{itemize}
	\item Uses cross-sectional data of eye disease progression (e.g. glaucoma).
	\item Build time series from cross-sectional data (``pseudo time series'')
	\item Subsample data; from each subsample, build a ``pseudo time series'' by finding the shortest path through the points
	\item Use these shortest paths as time series to train a time series model (i.e. a HMM model).
\end{itemize}

\item \fullcite{kafri2013dynamics}
\begin{itemize}
	\item Take snapshot of fixed cell population
	\item Fit 1D curve to trajectory (they assert that the two ``important'' parameters are levels of DNA and the level of ``anaphase-promoting complex'', so they project onto these two components and then fit a 1D closed curve to the data)
	\item Since the cells follow a closed cycle, given the closed curve, they can use mass balances to determine the relative rates/fluxes through the different stages
\end{itemize}

\item \fullcite{wyart2010evaluating}

\begin{itemize}
	\item Use RNA FISH technique to count levels of individual mRNAs in cells; requires cells to be fixed
	\item Propose models for dynamics of mRNA throughout cell cycle
	\item Show how they can use data from population of fixed cells to fit model (periodic model for level of mRNA production)
\end{itemize}

\item \fullcite{chen2012nonlinear}

\item \fullcite{das2013dynamic}

\item \fullcite{schindler2007inferring}
\begin{itemize}
	\item Want to order images that span many years
	\item Solve a constraint satisfaction problem (CSP)
	\item Find features in images, and then try to match up the features between the different images
	\item Use structure from motion (SFM) algorithms to construct 3D points and camera angles from 2D images
	\item Given the sets of features in each image, they then try to order the images in a way that is ``most consistent'' with how buildings are constructed in the real world (i.e. buildings are built, persist for some time, and then can be torn down)-- this is what they call the CSP
\end{itemize}
\item \fullcite{bigot2012geometric}

\item \fullcite{escudero2011epithelial}

\item \fullcite{machta2013parameter}

\item \fullcite{peng2013detecting}
\begin{itemize}
	\item Wants to find which genes/SNPs are responsible for variation in facial features
	\item Has DNA data and 3D face scans for many samples
	\item Looks at different facial models (all are based on landmarks)
	\item Tries to find which variations in the facial models correlate with which SNPs
\end{itemize}

\item \fullcite{gerber2010visual}
\begin{itemize}
	\item Want to find a ``good'' way to visualize high--dimensional functions
	\item First find local maxima and minima in data set
	\item Use these maxima and minima to segment the domain of the function into ``crystals''
	\item Approximate each crystal by a line/curve
	\item Embed the maxima and minima in 2D using PCA; draw lines connecting relevant maxima and minima
\end{itemize}

\item \fullcite{qiu2011discovering}
\begin{itemize}
	\item Want to order microarray data
	\item Look at specific (high--variance) genes (~3000 genes)
	\item Cluster genes into ``modules'' using k--means, so that similairly expressed genes are grouped together
	\item Within each module, construct a minimium spanning tree of the data (distance between nodes is Euclidean distance)
	\item Find set of MSTs that are most similar/consistent
	\item Assume these are the ``right'' MST and that they uncover the progression of the data
\end{itemize}

\item \fullcite{gupta2008extracting}
\begin{itemize}
	\item Want to temporally order gene expression data from cancer patients
	\item Use TSP to order gene expression data (can show that TSP path will recover the correct ordering under some probability model)
	\item Use a probabilistic gene/patient model to model the variation from patient to patient (model underlying dynamics with splines, and then assume rest in noise from patient-to-patient variability)
\end{itemize}

\item \fullcite{lau2009modeling}
\begin{itemize}
	\item Want to model trajectory/motion data where each sample is a trajectory
	\item Assume that the trajectories come from a second-order Markov model (i.e. $t+2$ depends on $t$ and $t+1$)
	\item Fit a Bayesian network that models the trajectory evolution using training data
	\item Can then produce new sample trajectories from this Bayesian network
\end{itemize}

\item \fullcite{toga2003temporal}
\begin{itemize}
	\item Want to model spatiotemporal changes in the 3D structure of the brain (so that one can detect disease, etc.)
	\item Model growth of different portions of the brain using deformation mapping (look at deformation of different regions of a {\em single} person's brain from one scan to the next)
	\item Can fit cortical models for each person in a population; can then look at variations across these models
\end{itemize}  

\item \fullcite{li2012modelling}
\begin{itemize}
	\item Want to use cross--sectional data to model disease progression
	\item Data are things that are known to be important/telling for that particlular disease (i.e. breast cancer data contains descriptors of cell morphology, Parkinson's disease data contains descriptors of speech patterns)
	\item Create ``pseudo time series'' by subsampling data, and from each subsample, finding shortest path that goes from a healthy sample to a diseased sample
	\item From these pseudo time series, can fit a hidden Markov model to find different states of disease progression and likely transitions from one state to another
\end{itemize}

\item \fullcite{denkinger2013set}
\begin{itemize}
	\item Want to develop a standardized data set that can be used to investigate temporal and causal neural processing
	\item Developed a set of 222 images which contain different causal and temporal relationships
\end{itemize}

\item \fullcite{xu2012comprehensive}
\begin{itemize}
	\item Want to quantify shape of liver lesions
	\item Take set of 14 common shape descriptors
	\item Use regression to find optimal combination of shape descriptors that can predict the similairity using a ``similairity reference standard''
	\item FINISH
\end{itemize}
\item \fullcite{sisan2010event}
\begin{itemize}
	\item Looks at time correlations in data to determine temporal ordering
	\item Goal is to determine in what order cellular events happen (i.e. does rxn A lead to rxn B, or does B lead to A)
	\item They look at the time-correlation between events.
	\item Asymmetry in time-correlation tells you in what order events happen (depending on model)
	\item Looking at time correlation over many samples integrates/averages out stocahsticity in system
\end{itemize}

\item \fullcite{golparvar2011monitoring}

\item \fullcite{golani1999phenotyping}
\begin{itemize}
	\item Want to construct video of construction progress from snapshots
	\item Building Information Models (BIM) 
\end{itemize}

\item \fullcite{anavy2014blind}

\item \fullcite{basu2014detecting}


\end{enumerate}


\section{Algorithms}

\begin{enumerate}

\item \fullcite{flusser2000independence}
\begin{itemize}
\item Construct moments (polynomials) of the pixel intensities that are invariant to rotations
\end{itemize}

\item \fullcite{besl1992method}
\begin{itemize}
	\item Iterative closest point (ICP) algorithm
	\item Aligns/registers point clouds
\end{itemize}

\item \fullcite{lowe1999object}
\begin{itemize}
	\item Scale-invartiant feature transform (SIFT)
	\item Computes feature vectors for image; features are scale, translation, and rotation invariant
\end{itemize}

\item \fullcite{sadler1992shift}
\begin{itemize}
	\item Calculate bispectrum for image
	\item Bispectrum is rotation-invariant
	\item Do not need to align the images
\end{itemize}

\item \fullcite{hilai1994recognition}
\begin{itemize}
	\item Construct rotation-invariant basis for images
\end{itemize}

\item \fullcite{ojala2002multiresolution}
\begin{itemize}
	\item Decribe local patches of image (texture) by binary code
	\item Define function on binary code that is rotation-invariant
	\item Classify using binary code
\end{itemize}

\item \fullcite{zhao2013fourier}
\begin{itemize}
	\item Construct PCA basis that is rotation-invariant
\end{itemize}

\end{enumerate}

\section{Applications}

\begin{enumerate}

\item \fullcite{hajnal2010medical}
\begin{itemize}
	\item Medical image registration (MRI, CT, etc.)
\end{itemize}

\item \fullcite{greenspan1994rotation}
\begin{itemize}
\item Texture recognition/classification
\item use Fourier basis; fit texture response (in Fourier basis) to rotations 
\end{itemize}

\item \fullcite{rowley1998rotation}
\begin{itemize}
	\item face recognition
	\item train neural net to estimate face orientation, then orient faces
\end{itemize}

\item \fullcite{simard1992efficient}
\begin{itemize}
	\item recognize digits
	\item only consider small rotations; fit plane to span of rotations of each digit, then see which plane new digit is closest to
\end{itemize}

\end{enumerate}


\end{document}