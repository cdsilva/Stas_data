\documentclass[12pt]{article}


\usepackage{biblatex}

\bibliography{references}

\begin{document}

\fullcite{wei2006conditional}

\fullcite{wei2006quantile}

\fullcite{li2009stochastic}

\fullcite{jordan2013behavioral}

\fullcite{tucker2010pseudotemporal}
\begin{itemize}
	\item Uses cross-sectional data of eye disease progression (e.g. glaucoma).
	\item Build time series from cross-sectional data (``pseudo time series'')
	\item Subsample data; from each subsample, build a ``pseudo time series'' by finding the shortest path through the points
	\item Use these shortest paths as time series to train a time series model (i.e. a HMM model).
\end{itemize}

\fullcite{kafri2013dynamics}
\begin{itemize}
	\item Take snapshot of fixed cell population
	\item Fit 1D curve to trajectory (they assert that the two ``important'' parameters are levels of DNA and the level of ``anaphase-promoting complex'')
	\item Since the cells follow a closed cycle, given the closed curve, they can use mass balances to determine the relative rates/fluxes through the different stages
\end{itemize}

\fullcite{wyart2010evaluating}

\fullcite{chen2012nonlinear}

\fullcite{das2013dynamic}

\fullcite{schindler2007inferring}

\fullcite{bigot2012geometric}

\fullcite{escudero2011epithelial}

\fullcite{machta2013parameter}

\fullcite{peng2013detecting}

\fullcite{gerber2010visual}

\fullcite{qiu2011discovering}

\fullcite{gupta2008extracting}

\fullcite{lau2009modeling}

\fullcite{toga2003temporal}
  
\fullcite{li2012modelling}

\fullcite{denkinger2013set}

\fullcite{xu2012comprehensive}

\fullcite{sisan2010event}

\fullcite{golparvar2011monitoring}

\fullcite{golani1999phenotyping}
\begin{itemize}
	\item Want to construct video of construction progress from snapshots
	\item Building Information Models (BIM) 
\end{itemize}

\fullcite{anavy2014blind}


\end{document}