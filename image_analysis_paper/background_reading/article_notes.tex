\documentclass[12pt]{article}


\usepackage[backend=bibtex]{biblatex}

\bibliography{references}

\begin{document}

\begin{enumerate}

\item \fullcite{wei2006conditional}
\begin{itemize}
		\item Want to develop a more informative growth chart metric
		\item Instead of looking at how atypical a current measurement is for an individual, they look at how atypical a time series is for an individual 
		\item This means that a person will be flagged if, for example, their weight changes dramatically in a short period of time, even if both measurements are within the ``normal'' range
		\item Fit model to data for ``normal'' time--series behavior
\end{itemize}

\item \fullcite{wei2006quantile}
\begin{itemize}
	\item Develop nonparameteric models (using spline regression) for growth charts
	\item Typical growth charts rely on normality assumptions (data is often transformed to fulfill such assumptions)
	\item Using nonparametric models can allow them to fit a much broader range of behaviors (i.e. bimodal distributions)
\end{itemize}

\item \fullcite{li2009stochastic}

\item \fullcite{jordan2013behavioral}
\begin{itemize}
	\item Look at phenotypic diversity of the swimming of the ciliate {\em Tetrahymena}
	\item Record the swimming trajectories of many samples over their lifetimes
	\item Define similarity metric between trajectories (compare distributions of speeds and angular velocities)
	\item Use multidimensional scaling (MDS) to embed the trajectories in a lower-dimensional space
	\item Find that there are two ``important'' dimensions that describe all trajectories
	\item Can then look at whether these (low-dimensional) phenotypes are preserved through generations
\end{itemize}

\item \fullcite{tucker2010pseudotemporal}
\begin{itemize}
	\item Uses cross-sectional data of eye disease progression (e.g. glaucoma).
	\item Build time series from cross-sectional data (``pseudo time series'')
	\item Subsample data; from each subsample, build a ``pseudo time series'' by finding the shortest path through the points
	\item Use these shortest paths as time series to train a time series model (i.e. a HMM model).
\end{itemize}

\item \fullcite{kafri2013dynamics}
\begin{itemize}
	\item Take snapshot of fixed cell population
	\item Fit 1D curve to trajectory (they assert that the two ``important'' parameters are levels of DNA and the level of ``anaphase-promoting complex'', so they project onto these two components and then fit a 1D closed curve to the data)
	\item Since the cells follow a closed cycle, given the closed curve, they can use mass balances to determine the relative rates/fluxes through the different stages
\end{itemize}

\item \fullcite{wyart2010evaluating}

\begin{itemize}
	\item Use RNA FISH technique to count levels of individual mRNAs in cells; requires cells to be fixed
	\item Propose models for dynamics of mRNA throughout cell cycle
	\item Show how they can use data from population of fixed cells to fit model (periodic model for level of mRNA production)
\end{itemize}

\item \fullcite{chen2012nonlinear}

\item \fullcite{das2013dynamic}

\item \fullcite{schindler2007inferring}
\begin{itemize}
	\item Want to order images that span many years
	\item Solve a constraint satisfaction problem (CSP)
	\item Find features in images, and then try to match up the features between the different images
	\item Use structure from motion (SFM) algorithms to construct 3D points and camera angles from 2D images
	\item Given the sets of features in each image, they then try to order the images in a way that is ``most consistent'' with how buildings are constructed in the real world (i.e. buildings are built, persist for some time, and then can be torn down)-- this is what they call the CSP
\end{itemize}
\item \fullcite{bigot2012geometric}

\item \fullcite{escudero2011epithelial}

\item \fullcite{machta2013parameter}

\item \fullcite{peng2013detecting}

\item \fullcite{gerber2010visual}

\item \fullcite{qiu2011discovering}

\item \fullcite{gupta2008extracting}

\item \fullcite{lau2009modeling}

\item \fullcite{toga2003temporal}
  
\item \fullcite{li2012modelling}

\item \fullcite{denkinger2013set}

\item \fullcite{xu2012comprehensive}

\item \fullcite{sisan2010event}
\begin{itemize}
	\item Looks at time correlations in data to determine temporal ordering
	\item Goal is to determine in what order cellular events happen (i.e. does rxn A lead to rxn B, or does B lead to A)
	\item They look at the time-correlation between events.
	\item Asymmetry in time-correlation tells you in what order events happen (depending on model)
	\item Looking at time correlation over many samples integrates/averages out stocahsticity in system
\end{itemize}

\item \fullcite{golparvar2011monitoring}

\item \fullcite{golani1999phenotyping}
\begin{itemize}
	\item Want to construct video of construction progress from snapshots
	\item Building Information Models (BIM) 
\end{itemize}

\item \fullcite{anavy2014blind}

\end{enumerate}



\end{document}