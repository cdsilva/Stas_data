\documentclass[11pt]{article}

\usepackage{amsmath}
\usepackage{amssymb}
\DeclareMathOperator*{\argmin}{arg\,min}

\begin{document}

\section{Techniques}

\subsection{Alignment Using Angular Synchronization}

Often when one collects images in experiments, each image is taken in a different orientation. 
%
Therefore, the first task in image analysis is to align the set of images. 
%
We will consider settings when alignment is done with respect to two--dimensional translations and rotations of the images, as each ``object of interest'' (i.e. each embryo) is in a different position and rotated at a different angle within the image frame. 

There are many ways to align images. 
%
``Template--based'' alignment \cite{ahuja2007template} is one technique, where one selects a specific image or {\em template}, and then tries to optimally align each image to that template.
%
However, this method can result in many misalignemnts for noisy images, and can be problematic if a ``good'' template image is not known {\em a priori}. 
%
Instead, we will use a technique called angular synchronization\cite{singer2011angular} to align sets of images.
%
Angular synchronization first computes the pairwise alignments for every possible pair of images, effectively using each image as a template for every other image.
%
From this large collection of pairwise alignments, it then tries to find optimal alignment for each image, such that these individual alignments are most consistent (in some metric) with the pairwise alignment.
%
Because we consider {\em all} pairwise alignments, angular synchronization is often much more robust to noise. 

We will first discuss angular synchronization with {\em only} rotations. 
%
Let $I_1, \dots, I_m \in \mathbb{R}^{n \times n}$ denote the (grayscale) images that we wish to align with respect to rotations.
%
We assume that each image $I_i$ is a noisy rotated copy of the underlying image $I_{true}$ (which we are {\em not} given), such that 
\begin{equation}
I_i = R_i I_{true} + \xi_i,
\end{equation}
where $R_i \in \mathbb{R}^{2 \times 2}$ is an unknown rotation matrix and $\xi_i$ is random (often Gaussian) noise \footnote{The rotation matrices $R_i$ do not operate directly on the $n \times n$ image matrix $I_{true}$; however, we feel that this abuse of notation makes our description of the problem more intuitive.}. 
%
Our goal is to obtain estimates of $R_1, \dots, R_m$ so that we can rotate the images $I_1, \dots, I_m$ to be aligned.
%
Up to noise, 
\begin{equation} \label{eq:pairwise_rot}
I_i \approx R_i R_j^T I_j;
\end{equation}
 note that \eqref{eq:pairwise_rot} does not require knowledge of the true underlying image $I_{true}$.
%
We can obtain {\em estimates} of $R_i R_j^T$ by computing the rotations to optimally align $I_j$ to $I_i$. 
%
Let $R_{ij}$ denote the rotation that optimally aligns $I_j$ to $I_i$.
%
We then construct the matrix $H \in \mathbb{R}^{2m \times 2m}$, where $H$ is an $m \times m$ matrix of $2 \times 2$ blocks with
\begin{equation} \label{eq:H_to_R}
H_{ij} = R_{ij}
\end{equation}
%
We note that, under our assumptions, $H_{ij} \approx R_i R_j^T$, and so
\begin{equation} \label{eq:H_low_rank}
	H \approx 
	\begin{bmatrix}
	R_1 \\
	R_2 \\
	\vdots \\
	R_m
	\end{bmatrix}
	\begin{bmatrix}
	R_1^T R_2^T \dots R_m^T
	\end{bmatrix}
\end{equation}
%
We can (approximately) recover $R_1, R_2, \dots, R_m$ by computing the top block eigenvector of $H$ (i.e the eigenvectors corresponding to the two largest eigenvalues, stacked into a $2m \times 2$ matrix), and register the images by rotating image $i$ by $R_i^T$. 
%
Furthermore, this eigendecomposition also considers {\em higher--order} consistency information. 
%
For example, given our pairwise estimates $R_{ij}$, we know that relationships of the form
\begin{equation} \label{eq:triplet_consistency}
R_{ik} R_{kj} \approx R_i R_k^T R_k R_j^T = R_i R_j^T
\end{equation}
should also hold.
%
However, 
\begin{equation}
(H^2)_{ij} = \sum_k R_{ik} R_{kj}
\end{equation}
and so {\em all} infomation of the form in \eqref{eq:triplet_consistency} is contained in the matrix $H^2$.
%
Because $H$ and $H^2$ have the same eigenvectors, our problem formulation accounts for not only pairwise alignment information, but also these higher--order considerations. 

We now note that, for \eqref{eq:H_low_rank} to be satisfied, we require that the group operation matrices be orthogonal, i.e. $R_i^{-1} = R_i^T$. 
%
The typical matrix representation for $ISO(2)$, the group of two--dimensional translations and rotations (using the matrix representation for the affine group \cite{...}), does not satisfy this property. 
%
However, we can (approximately) represent rotations and translations in two--dimensions using three--dimensional rotation matrices.
%
We do this by projecting the image onto a portion of the surface of a (three--dimensional) sphere.
%
Rotations in two--dimensions now correspond to rotations around one principal axis in three--dimensions, and translations in two--dimensions correspond (approximately) to rotations around the other two principal axes in three--dimensions \footnote{Clearly, not all translations can be described this way, as translations can range from $-\infty$ to $+ \infty$, and rotations are only from $0$ to $2 \pi$. However, the images we will be considering are already mostly centered, and so we are only interested in small translations that are well within the $[0, 2\pi)$ range.}.
%
These rotation matrices are (by definition) orthoganol, and successive applications of various translations and rotations can be described via multiplication of the corresponding rotation matrices in $SO(3)$.

The first step in angular synchronization is to compute the translations and rotations to optimally align each pair of data points. 
%
For each image pair $I_i$ and $I_j$, we solve
\begin{equation}\label{eq:opt_pairwise}
(\theta_{ij}, dx_{ij}, dy_{ij}) = \argmin_{
\begin{matrix}
\theta \in [0, 2\pi) \\ 
dx \in [-\Delta, \Delta]\\ 
dy \in [-\Delta, \Delta]
\end{matrix}
} \|f(I_j, \theta, dx, dy) - I_i \|_F^2.
\end{equation}
where $\| \cdot \|_F$ denotes the Frobenius norm, $\Delta$ is the maximum number of pixels by which we are allowed to translate the image, and $f(I_j, \theta, dx, dy)$ is a function that first rotates the image $I_j$ by $\theta$ degrees, then translates the image by $dx$ pixels in the $x$ direction, and finally translates the image by $dy$ pixels in the $y$ direction. 
%
We then compute the three--dimensional rotation, $R_{ij} \in SO(3)$, from the two--dimensional translation and rotation parameters $\theta_{ij}$, $dx_{ij}$, and $dy_{ij}$ (see Appendix for more details).
%
We then build the matrix $H \in \mathbb{R}^{3m \times 3m}$ as described in \eqref{eq:H_to_R}, and compute the top block eigenvector of $H$, $\hat{R} \in \mathbb{R}^{3m \times 3}.$ 
%
The $3 \times 3$ blocks of $\hat{R}$ contain approximations of $R_1, \dots, R_m$.  
%
Because the blocks are not guaranteed to lie in $SO(3)$, we then project each block back to $SO(3)$ (see Appendix).
%
To register the images, we compute the translations and rotation corresponding to $R_i^T$. 

\subsection{Ordering Using Diffusion Maps}

Given our set of aligned images, we now wish to order them {\em in time} so that we can reconstruct the developmental dynamics.
%
We will use diffusion maps (DMAPS) \cite{coifman2005geometric} to temporally order our data.
%
DMAPS aims to uncover a parameterization of high-dimensional data sampled from a low-dimensional nonlinear manifold.
%
The {\em essential} requirement for DMAPS is an appropriate distance metric $d(I_i, I_j)$ for comparing data points. 
%
This can as simple as the standard Euclidean distance, or a more complex metric (such as a distance between features of the data points) for other data sets.

In our framework, we assume that our data lie on a one--dimensional curve or manifold, and that this one dimension is correlated with time. 
%
We will uncover a parameterization of this curve by only considering {\em local} neighbors of each data point.
%
To illustrate how local neighbors can allow us to uncover a global parameterization, assume we have data $x_1, x_2, \dots, x_m \in \mathbb{R}$ such that $x_1 < x_2 < \dots < x_m$.
%
We consider $x_i$ ``close to'' $x_{i+1}$ and $x_{i-1}$, and $x_i$ ``far away from'' all other data points. 
%
We then construct the weight matrix, $W \in \mathbb{R}^{m \times m}$, where $W_{ij}$ is large if $x_i$ is close to $x_j$,
\begin{equation}
W = 
\begin{bmatrix}
	1 & 1/2 & 0 & 0 & \dots & 0 \\
	1/2 & 1 & 1/2 & 0 & \dots & 0 \\
	\ddots & \ddots & \ddots & \ddots & \ddots & \ddots \\
	0 & 0 & 0 & \dots & 1 & 1/2 \\
	0 & 0 & 0 & \dots & 1/2 & 1 
\end{bmatrix}.
\end{equation}
%
If we normalize this matrix such that $\sum_j W_{ij} = 1$, then one can show that the first eigenvector is the all--ones vector, $\phi_1 = [1 1 \dots 1]^T$, and the second eigenvector, $\phi_2 = [cos(\pi/m) \cos(2 \pi/ m) \dots \cos((m-1) \pi / m) \cos(\pi)]^T$.
%
Therefore, the second eigenvector, $\phi_2$ is {\em one--to--one} with the parameterization of the data $x_1, x_2, \dots, x_m$. 
%
Therefore, sorting our data by the value of the entries in $\phi_2$ will allow us to order our data.

For the above analysis, the key step was having a notion of ``closeness'' between data points so that we can construct the matrix $W$.
%
For general data in high--dimensional space, we typically use a Gaussian kernel,
\begin{equation} \label{eq:dmaps_W}
W_{ij} = \exp \left( -\frac{d^2(x_i, x_j)}{\epsilon^2} \right)
\end{equation}
and $\epsilon$ is a characteristic distance between data points.
%
Therefore, points closer than $\epsilon$ apart are considered ``close'' and points farther than $\epsilon$ apart are considered ``far away''.
%
$\epsilon$ can be chosen using several techniques (see, for example \cite{coifman2008graph}); in practice, we choose $\epsilon$ to be the median of the pairwise distances between data points.

We then compute the diagonal matrix $D$, where $D_{ii} = \sum_{j=1}^{m} W_{ij}$, and the matrix $A$, where
\begin{equation} \label{eq:dmaps_A}
A = D^{-1} W.
\end{equation} 
%
We calculate the eigenvectors $\phi_1, \phi_2, \dots, \phi_m$ and eigenvalues $\lambda_1, \lambda_2, \dots, \lambda_m$ and order them such that $|\lambda_1| \ge |\lambda_2| \ge \dots \ge |\lambda_m|$ \footnote{Because the matrix $A$ is similar to the symmetric matrix $D^{-1/2} W D^{-1/2}$, $A$ is guaranteed to have real eigenvalues and real, orthogonal eigenvectors.}. 
%
Because the matrix $A$ is row-stochastic, $\lambda_1=1$ and $\phi_1$ is a constant vector.
%
In general, the next few eigenvectors $\phi_2, \dots, \phi_m$ give ``meaningful'' embedding coordinates for the data, such that $\phi_j(i)$ gives the $j^{th}$ embedding coordinate of the $i^{th}$ data point. 
%
In our setup, we assume that our data are inherently one--dimensional, and that this one dimension is correlated with time.
%
Therefore, ordering our data by $\phi_2(j)$ will allow us to automatically order our data in time. 

\subsection{Vector Diffusion Maps}

Many times, we would to both align and order our data.
%
Our proposed alignment methodology, angular synchronization, utilizes the eigendecompostion of the matrix $H \in \mathbb{R}^{3m \times 3m}$, and our proposed ordering algorithm, diffusion maps, utilizes the eigendecomposition of the matrix $A \in \mathbb{R}^{m \times m}$.
%
We can combine the two steps into one eigencomputation that allows us to {\em simultaneously} recover the optimal alignments and the temporal ordering.
%
This technique is called vector diffusion maps \cite{singer2012vector}.
%
It not only reduces the number of required eigencomputations, but also is much more robust to noise, as mistakes in alignment are also considered when ordering the data.

For vector diffusion maps, we construct the matrix $S \in \mathbb{R}^{3m \times 3m}$, with
\begin{equation}
	S_{ij} = A_{ij} H_{ij}
\end{equation}
%
where $A$ is defined in \eqref{eq:dmaps_A}, the distance metric $d(I_i, I_j)$ is the Frobenius norm between the images {\em after} optimal pairwise alignment ($d(I_i, I_j) = \| f(I_j, \theta_{ij}, dx_{ij}, dy_{ij}) - I_i \|_F$), and $H$ is defined in \eqref{eq:H_to_R}.

We then compute the eigenvalues $\lambda_1, \lambda_2, \dots, \lambda_{3m}$ and eigenvectors $\phi_1, \phi_2, \dots, \phi_{3m}$ of $S$, and order them such that $|\lambda_1| \ge |\lambda_2| \ge \dots \ge |\lambda_{3m}|$.
%
Again, the top (block) eigenvector of $S$ gives us approximations to the optimal rotations for each image.
%
However, we now also obtain embedding coordinates for our images.
%
In general, the embedding coordinates are given by 
\begin{equation}
\psi_{k,l} (i) = \langle \phi_k(i), \phi_l(i) \rangle
\end{equation}
where $\phi_k(i) \in \mathbb{R}^3$ denotes the $i^{th}$ block of $\phi_k$.
%
If we assume that our data are one--dimensional, and that the rotations and the dynamics are uncoupled and therefore separable, one can show that the ``correct'' embedding coordinate (i.e. the analog of $\phi_2$ from the diffusion maps case) is $\psi_{1,4}$.
%
If no such claims can be made, then one often runs a second step of diffusion maps, using $\psi_{k,l}$ as the data points, in order to uncover the true manifold modulo symmetries. 


\section{Appendix}

\subsection{Angular Synchronization Implementation}

\subsubsection{Computing Optimal Pairwise Alignments}

In \eqref{eq:opt_pairwise}, we take $\Delta=20$, which corresponds to a 20\% shift in the image. 
%
The solution to \eqref{eq:opt_pairwise} is not easily computed, as the objective function will most likely be nonconvex.
%
Therefore, instead of using an optimization procedure to compute the solution to \eqref{eq:opt_pairwise}, we choose to discretize the search space and exhaustively search to estimate the solution to \eqref{eq:opt_pairwise}.
%
We discritize the search space of rotations into 20 possible rotations ($d\theta  \in \{0, \pi/10, \pi/5, \dots, 9 \pi/5, 19\pi/10 \}$), and 11 possible translations in both the $x$ and $y$ directions ($dx, dy \in \{-20, -16, -12, \dots, 12, 16, 20 \}$). 
%
We then check all possible combinations for the best rotation and translations that align $I_j$ to $I_i$. 
%
Although this can be somewhat time intensive, it is not prohibitive for the size of data sets which we are considering, and can be trivially parallelized for larger data sets if necessary.
%
The solution to this search will not be the exact solution to \eqref{eq:opt_pairwise}, but it will (most likely) be a close approximation.
%
Since our techniques are robust to noise, close approximations will be sufficient to obtain accurate results.

\subsubsection{Converting from $ISO(2)$ to $SO(3)$}

In the space of three--dimensional rotations, the Euler angles $\alpha_{ij}$, $\beta_{ij}$, and $\gamma_{ij}$ correspond to
\begin{equation} \label{eq:angle_relations}
\begin{aligned}
	\alpha_{ij} &= \theta_{ij} \\
	\beta_{ij} &= \frac{dx_{ij}}{n} \times \eta_{proj} \\
	\gamma_{ij} &= \frac{dy_{ij}}{n} \times \eta_{proj} \\
\end{aligned}
\end{equation}
where $\eta_{proj}$ is the angular portion of the sphere onto which we choose to project the image.
%
We take $\eta_{proj} =  \pi/8$, so the image lies on a $\pi/8 \times \pi/8$ radians portion of the unit sphere in $\mathbb{R}^3$.
%
From here, we can write rotations around the three principal axes in terms of the three Euler angles
\begin{equation}
\begin{aligned}
	R^x_{ij} &= \begin{bmatrix}
	1 & 0 & 0 \\
    0 & \cos(\alpha) & -\sin(\alpha) \\
    0 & \sin(\alpha) & \cos(\alpha)
	\end{bmatrix} \\
	R^y_{ij} &= \begin{bmatrix}
	\cos(\beta) & 0 & \sin(\beta) \\
    0 & 1 & 0 \\
    -\sin(\beta) & 0 & \cos(\beta)
    \end{bmatrix} \\
	R^z_{ij} &= \begin{bmatrix} 
	\cos(\gamma) & -\sin(\gamma) & 0 \\
    \sin(\gamma) & \cos(\gamma) & 0 \\
    0 & 0 & 1 
    \end{bmatrix}
\end{aligned}
\end{equation}
where $R^x_{ij}$, $R^y_{ij}$, and $R^z_{ij}$ correspond to rotations around the $x$, $y$, and $z$ axes, respectively.
%
We then write the total rotation $R_{ij} \in SO(3)$ as 
\begin{equation} \label{eq:total_rot}
	R_{ij}	 = R^z_{ij} \times R^y_{ij} \times R^x_{ij}
\end{equation}
%
\eqref{eq:total_rot} corresponds to first rotating the image by $\theta$, then translating the image by $dx$ pixels in the $x$ direction, and finally translating the image by $dy$ pixels in the $y$ direction (we note that, because the rotation matrices operate from the left, the rightmost rotation matrix in the product in \eqref{eq:total_rot} corresponds to the first operation we perform on the image).

\subsubsection{Computing Optimal Translations and Rotations from the Eigenvectors of $H$}

From the top block eigenvector $\hat{R} \in \mathbb{R}^{3m \times 3}$, we would like to compute the optimal translation and rotation for each image. 
%
The optimal three--dimensional rotations are given by $R_i = U_i V_i^T$, where $U_i$ and $V_i$ are the left and right singular vectors of $\hat{R}_i$, respectively \cite{...}. 
%
Because the eigenvectors of $H$ are determined up to a sign, we adjust the signs of the eigenvectors so that $det(R_i) = +1$ to ensure proper rotations.

From the matrices $R_1, \dots, R_m$, we now wish to find the corresponding translations and rotations of the images.
%
Because we are permitted to adjust our rotations by a global rotations, we first multiply all rotations by $R_1^T$ to ensure that all the images are (approximately) in the region of the sphere where we began.
%
We then compute the Euler angles $\alpha$, $\beta$, and $\gamma$ from the rotation matrix $R$ using the following relationships
\begin{equation}
\begin{aligned}
R_{1,1} & = \cos(\beta)\cos(\gamma) \\
R_{2,1} & = \cos(\beta)\sin(\gamma) \\
R_{3,1} & = -\sin(\beta) \\
R_{3,2} & = \sin(\alpha)\cos(\beta) \\
R_{3,3} & = \cos(\alpha)\cos(\beta) 
\end{aligned}
\end{equation}
%
We can then invert \eqref{eq:angle_relations} to compute the optimal translation and rotation for each image from the Euler angles.

\bibliographystyle{plain}
\bibliography{../../references/references}

\end{document}